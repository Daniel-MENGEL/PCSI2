\input ../../../Modeles/TP.txt

\def\nompdf{Calorimetrie}



% capacités thermiques : kJ.kg/K
% eau 4,18
% Cu 0,385
% Fe 0,444
% Al 0,897

\begin{document}


\title{Calorimétrie}

\maketitle{Séance de TP \hfill n°7}

%\rapport

\paragraph{Introduction :} la calorimétrie a pour objet l'étude expérimentale des échanges thermiques entre des corps (en général des phases condensées).\\
On emploiera ici deux types de méthodes : la méthode des mélanges et la méthode électrique.

\objectifs
\begin{list}{\textbullet}{}
\item Manipuler un calorimètre et ses accessoires,
\item Savoir écrire et exploiter un bilan enthalpique,
\item Connaître différentes méthodes de détermination de capacités thermiques et chaleurs latentes de changement d'état,
\item Déterminer l'incertitude (écart) type $u$ sur les valeurs mesurées.
\end{list}

\materiel
\begin{list}{\textbullet}{}
\item Sur chaque paillasse : un calorimètre et ses accessoires, deux bechers dont au moins un en pyrex, une plaque chauffante, un thermomètre, un résistor, une alimentation stabilisée, un multimètre, un chronomètre et un chiffon. Un ordinateur sur lequel Anaconda est installé pour utilisation de \href{https://colab.research.google.com/drive/1PetqwzUOcJiuxUtyVSsJfLvtuIA2niTm#scrollTo=tA9XPTL0j2ae}{Notebook Jupyter}
\item Au bureau : de l'eau liquide, de la glace fondante, une balance, une (ou deux) bouilloire électrique. Des échantillons de cuivre, fer et aluminium.
\end{list}


\paragraph{Précautions :}
\begin{list}{\Lightning}{}
\item prenez garde à ne pas vous bruler sur la plaque chauffante ou en manipulant de l'eau bouillante ou un métal chauffé,
\item couvrez bien le calorimètre et effectuez rapidement les mesures de température pour limiter les fuites thermiques, manipulez ``rapidement'' la glace.
\end{list}


\indications
\begin{list}{\textbullet}{}
\item On notera $c_{e}$ la capacité thermique massique de l'eau, $c_\text{métal}$ celle des différents métaux (Cu, Fe ou Al) mis à votre disposition
et $\Delta h_\text{fusion}$ l'enthalpie massique de fusion de la glace.
\item On appelle $\mu=\frac{C_{cal}}{c_{e}}$ la masse en eau du calorimètre où $C_{cal}$ est la capacité thermique du calorimètre.
\end{list}


\questions en détaillant le protocole utilisé et les sources d'erreur sur les mesures effectuées, déterminer les valeurs suivantes.
\begin{list}{\textbullet}{}
\item La masse en eau $\mu$ de votre calorimètre.
\item La capacité thermique massique de l'eau $c_{e}$.
\item La capacité thermique massique d'un métal $c_\text{métal}$.
\item L'enthalpie massique de fusion de la glace $\Delta h_\text{fusion}$.
\end{list}


\bigskip




\tdm

%\bigskip

\nbrepostesetduree{10}{4 h}


\end{document}
